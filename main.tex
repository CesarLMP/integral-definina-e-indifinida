\documentclass{article}
\usepackage{graphicx} 
\usepackage[spanish] {babel}
\usepackage{amsthm}
\usepackage{amssymb}
\theoremstyle{plain}
\newtheorem{theorem} {Teorema} [section]
\newtheorem{corollary}{Corolario}[theorem]
\newtheorem{lemma}[theorem]{Lema}
\usepackage{framed}

\title{tarea com-117}
\author{Cesar Lucas Mamani Posto}
\date{September 2024}

\begin{document}

\maketitle
\begin{center}
   Integral Definida
\end{center}


\begin{framed}
\begin{theorem}
\begin{enumerate}
    \item $\int_a^b f(x) dx = \int_a^b f(s) ds$ (variable muda)
    \item Si $b > a$, entonces $\int_a^b f(x) dx = -\int_b^a f(x) dx$
    \item Si $f(a)$ existe, entonces $\int_a^a f(x) dx = 0$
\end{enumerate}
\end{theorem}
\end{framed}

\begin{center}
    \includegraphics[scale=0.1]{2.jpg}
\end{center}


Como es evidente, la integral definida resuelve el problema del área bajo la curva; sin embargo, es muy importante aclarar que la integral definida se utiliza para muchas otras aplicaciones físicas, económicas y matemáticas

\begin{framed}
\begin{theorem}
Si $f$ es una función integrable y $f(x) \geq 0$ para todo $x$ en $[a,b]$, entonces el área bajo la curva trazada por $f$ entre $a$ y $b$ es
\begin{center}
    $A = \int_a^b f(x) \, dx$
\end{center}
\end{theorem}
\end{framed}
\begin{center}
    \includegraphics[scale=0.5]{1.jpg}
\end{center}

No todas las funciones son integrables, pero las continuas sí lo son

\begin{framed}
\begin{theorem}
    Si $f$ es continua en $[a,b]$, entonces $f$ es integrable en $[a,b]$
\end{theorem}
\end{framed}


\begin{framed}
\begin{theorem}
    $\int_a^b c dx = c(b - a)$ donde $c$ es una constante
\end{theorem}
\end{framed}


\begin{framed}
\begin{theorem}
Si $f$ es integrable en $[a,b]$ y $c$ es un número real arbitrario, entonces $cf$ es integrable en $[a,b]$ y 
\begin{center}
    $\int_a^b cf(x) dx = c \int_a^b f(x) dx$
\end{center}
\end{theorem}
\end{framed}


\begin{framed}
\begin{theorem}
Si $f$ y $g$ son integrables en $[a,b]$, entonces $f+g$ y $f-g$ son integrables en $[a,b]$ y 
\begin{center}
    $\int_a^b (f(x) \pm g(x)) \, dx = \int_a^b f(x) \, dx \pm \int_a^b g(x) \, dx$
\end{center}
\end{theorem}
\end{framed}

\begin{framed}
\begin{theorem}
Si $a < c < b$ y $f$ es integrable en $[a,c]$ y en $[c,b]$, entonces $f$ es integrable en $[a,b]$ y 
\begin{center}
    $\int_a^b f(x) dx = \int_a^c f(x) dx + \int_c^b f(x) dx$
\end{center}
\end{theorem}
\end{framed}

\begin{framed}
\begin{theorem}
Si $f$ es integrable en $[a,b]$ y $f(x) \geq 0$ para todo $x$ en $[a,b]$, entonces 
\begin{center}
    $\int_a^b f(x) dx \geq 0$
\end{center}
\end{theorem}
\end{framed}



\end{document}
